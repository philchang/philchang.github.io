\documentclass[12pt]{article}
\usepackage[landscape]{geometry}
\usepackage{graphicx,color} 
\include{def.tex}


\begin{document}
\pagestyle{empty}
\noindent Consider a charge $q$ at $x=-a$ and a charge $-2q$ at $x=a$.  How much work does it take to bring a charge $+q$ to $x=0$ from infinity.  

\newpage
\noindent A stationary ring of radius, $R$, has total charge $Q$.  A small particle of charge $-q$ is constrained to move along the symmetry axis of the ring, which is the x-axis.  (a) find the potential, $V$ of the particle as a function of x (b) Show that for small $x$, the potential has the form $V(x) \approx V(0) + \alpha x^2$ and find $\alpha$, which is a constant. 

\newpage
\noindent Consider a charge $q$ that is enclosed by a spherical conducting shell of radius $R$ that is grounded.  What is the charge on the shell? 

%\newpage
%\noindent Consider the following model the hydrogen atom.  A small point change of $+q$ is surrounded by a spherical distribution of negative charge of the form $\rho = \rho_0 \exp(-r/r_0)$.  Using the fact that hydrogen atom is neutral, find $\rho_0$ in terms of $q$ and $r_0$.  Find the potential as a function of $V(r)$.

\newpage 
\noindent Consider a uniformly charged sphere of radius $R$ and total charge $Q$.  How much energy does it take to assemble this sphere?   
\end{document}
