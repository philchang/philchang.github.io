\documentclass[12pt]{article}
\usepackage{times}
\usepackage{epsfig}
\usepackage{natbib}
\usepackage{caption}
\usepackage{multicol,wrapfig}
%\usepackage{deluxetable}
\usepackage{graphics,graphicx}
\usepackage{rotating}
\usepackage{amssymb}
\usepackage{amsmath,fancyhdr}
%\usepackage{tweaklist}

\captionsetup{labelfont=bf,font=normal,indention=0.05cm}


\makeatletter
\renewcommand{\section}{\@startsection%
{section}{1}{0mm}{-0.5\baselineskip}%
{0.2\baselineskip}{\normalfont\large\bfseries}}%
\renewcommand{\subsection}{\@startsection%
{subsection}{1}{0mm}{-0.2\baselineskip}%
{0.2\baselineskip}{\normalfont\bfseries}}%
\makeatother

\setlength{\textwidth}{6.5in} 
\setlength{\textheight}{9in}
\setlength{\topmargin}{0in} 
\setlength{\oddsidemargin}{0in}
\setlength{\evensidemargin}{0in} 
\setlength{\headheight}{0in}
\setlength{\headsep}{0in} 
\setlength{\hoffset}{0in}
\setlength{\voffset}{0in}
\setlength{\textfloatsep}{0.0in}
\setlength{\floatsep}{0.0in}


%\renewcommand{\itemhook}{\setlength{\topsep}{0pt}%
%  \setlength{\itemsep}{-5pt}}


\pagestyle{fancy}
\lhead{}
\chead{}
\rhead{}
\cfoot{\thepage}
\lfoot{}
\rfoot{}
\renewcommand{\headrulewidth}{0.0pt}
\renewcommand{\footrulewidth}{0.0pt}

%\usepackage[dvips]{graphicx}
%\usepackage{emulateapj5}

\renewcommand{\rmdefault}{ptm}

\newcommand{\vect[1]}{\bmath{#1}}
\newcommand{\ihat}{\hat{i}}
\newcommand{\jhat}{\hat{j}}
\newcommand{\khat}{\hat{k}}


\begin{document}
\title{The Plasma Physics of TeV Blazars}

\centering{Physics 209: Problem Set 3} 
\\
\centering{Due Date: September 23, 2014}
\bigskip
\begin{enumerate}

\item (10 pts) Suppose you subject a particle to an acceleration of $\vec{a} = 4 \ihat + 3 \jhat$ and at $t=0$, the velocity and position is zero. (a) Find $\vec{v}(t)$ and $\vec{r}(t)$. (b) Find the equation of the path of the particle in the x-y plane. 

\item (10 pts) In a famous scene, Thelma and Louis drive a car of a cliff. If the cliff is 30 m high and their car is traveling at 20 m/s, (a) how far from the cliff would they hit the ground.  (b) what is the magnitude of their velocity when they hit the ground? 

\item (10 pts) TM Ch 3, problem 72 

\item (10 pts) A cannon on level ground is aimed at 60 degrees above the horizon and fires a shell with a velocity of 300 m/s.  (a) How far downrange does the shell hit? (b) Show that this range is the same if the cannon was aimed at 30 degrees (c) For what angle is the time of flight greater?

\item (20 pts) TM Ch 3 Problem 97

\item (20 pts) A cannon that is aim 60 degree above the horizon fires at a object that is 1000m downrange and moving away from the cannon at 20 m/s.  How fast does the shell have to leave the cannon in order to hit the object?  At what position from the cannon does shell hit the target?     

\item (10 pts) A 1000 kg car zooms around a circular bend with a radius of 100m at 20 m/s? What direction is the force in on the car?  What is the magnitude of this force? Assume that the car remains at constant speed. 

\item (10 pts) TM Ch 3, problem 69

\item This question is for statistics only.  Did you view video tutorials \\
(http://www.gravity.phys.uwm.edu/~pchang/Site/phys209.html) associated with this problem set (Y/N)? If so how much did they help you complete the problem set (1 - not useful to 10 - extremely useful)?
\end{enumerate}


\end{document}

