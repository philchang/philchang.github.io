\documentclass[12pt]{article}
\usepackage{times}
\usepackage{epsfig}
\usepackage{natbib}
\usepackage{caption}
\usepackage{multicol,wrapfig}
%\usepackage{deluxetable}
\usepackage{graphics,graphicx}
\usepackage{rotating}
\usepackage{amssymb}
\usepackage{amsmath,fancyhdr}
%\usepackage{tweaklist}

\captionsetup{labelfont=bf,font=normal,indention=0.05cm}


\makeatletter
\renewcommand{\section}{\@startsection%
{section}{1}{0mm}{-0.5\baselineskip}%
{0.2\baselineskip}{\normalfont\large\bfseries}}%
\renewcommand{\subsection}{\@startsection%
{subsection}{1}{0mm}{-0.2\baselineskip}%
{0.2\baselineskip}{\normalfont\bfseries}}%
\makeatother

\setlength{\textwidth}{6.5in} 
\setlength{\textheight}{9in}
\setlength{\topmargin}{0in} 
\setlength{\oddsidemargin}{0in}
\setlength{\evensidemargin}{0in} 
\setlength{\headheight}{0in}
\setlength{\headsep}{0in} 
\setlength{\hoffset}{0in}
\setlength{\voffset}{0in}
\setlength{\textfloatsep}{0.0in}
\setlength{\floatsep}{0.0in}


%\renewcommand{\itemhook}{\setlength{\topsep}{0pt}%
%  \setlength{\itemsep}{-5pt}}


\pagestyle{fancy}
\lhead{}
\chead{}
\rhead{}
\cfoot{\thepage}
\lfoot{}
\rfoot{}
\renewcommand{\headrulewidth}{0.0pt}
\renewcommand{\footrulewidth}{0.0pt}

%\usepackage[dvips]{graphicx}
%\usepackage{emulateapj5}

\renewcommand{\rmdefault}{ptm}

\newcommand{\vect[1]}{\bmath{#1}}
\newcommand{\ihat}{\hat{i}}
\newcommand{\jhat}{\hat{j}}
\newcommand{\khat}{\hat{k}}


\begin{document}
\title{The Plasma Physics of TeV Blazars}

\centering{Physics 209: Worksheet 3} 
\\
\bigskip
\begin{enumerate}

\item A probe in deep space moves with a constant velocity of $\vec{v} = (10
\hat{x} + 15 \hat{y} )$ m/s.  It fires its engines for 100 s and
experiences constant acceleration.  After it stops its engines, the
probe moves with a constant velocity of $\vec{v} = (20 \hat{x} + 25
\hat{y} )$ m/s.  What acceleration did the probe experience?

\item Suppose you subject a particle with a velocity of $\vec{v} = 4 t
  \ihat + 3 \jhat$ and at $t=0$, the position is zero. (a) Find the
  acceleration and $\vec{r}(t)$. (b) Find the equation of the path of
  the particle in the x-y plane.

\item In class we showed that a can will be hit as it drops at the
  same time that a projectile is initially launch at it.  Suppose that
  the can is 10 m (horizontally) away from the cannon and is 5 m above
  the table.  If the cannon is on the ground and aim at 45 degrees
  above the horizon, what is the minimum speed that the shell has to
  leave the cannon to hit the can before it hits the ground?

\item Three swimmers run off a 20 m high cliff simultaneously at a speed of 3 m/s, 4 m/s, and 5 m/s.  When do they each hit the water?  How far are they from the cliff when they do?

\item Show that the maximum range that you can reach for fixed initial
  velocity is obtain when you train the cannon at a 45 degree angle
  above the horizon.  

\item  An archer tries to hit a target that sits 20 m high on a castle
  wall that is 100 m away.  If the archer aims at 30 degrees above the
  horizon, what speed must the arrow leave to hit this target.

\item A sports car zooms around a 100m circular radius turn and feels
  an acceleration of 5 m/s$^2$.  What direction is the acceleration?
  What is the speed of the car?

\item Suppose the turn above in increased to 200m, but the speed is
  doubled as well.  By what factor does the acceleration increase or
  decrease? 

\end{enumerate}


\end{document}

