\documentclass[12pt]{article}
\usepackage[landscape]{geometry}
\usepackage{graphicx,color} 
\include{def.tex}


\begin{document}
\pagestyle{empty}
\noindent Consider the pulley system below:  How much velocity does the 40 kg mass attain when it falls down 1 m and the 30 kg mass falls up 1 m.\\
\resizebox{!}{6cm}{\input pulley.pstex_t}
\newpage
\noindent Consider a rigid pendulum below where the pivot is frictionless and the rod is massless.  What the the velocity of the mass when it reaches the lowest point?\\
\resizebox{12cm}{!}{\input pendulum.pstex_t}
\newpage
\noindent A spring with a spring constant of 1000 N/m is attached to a 10 kg mass and is stretched 0.5 m pass its equilibrium point.  If the coefficient of 
kinetic friction is 0.3, (a) what is the speed of the mass as the spring moves past is equilibrium point?  (b) how much work is done by the force of friction?
\end{document}
