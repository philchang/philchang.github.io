\documentclass[12pt]{article}
\usepackage[landscape]{geometry}
\usepackage{pdflscape}
\usepackage[pdftex]{graphicx,color} 
\include{def.tex}
\newcommand{\ihat}{\hat{i}}
\newcommand{\jhat}{\hat{j}}
\newcommand{\khat}{\hat{k}}


\begin{document}

\noindent Two identical solid rotating disks around the same axis, but with two different rotation rates: $\omega_1 = -10$ rad/s, $\omega_2 = 10$ rad/s are forced together and rotate constantly.  What is the final rotation rate of the two disks? 

\newpage
\noindent On a spinning 100 kg disk with $\omega = 10$ rad/s and radius 1 m, I drop a 10 kg pellet on its rim.  What is its new angular speed?

\newpage
\noindent A 1 kg pellet hits a 10 kg rod at one end that is 1 m long that swings on one end.  If the 0.1 kg pellet has a velocity of 1 m/s what is the initial rotation rate of the rod?
\newpage
\noindent A large pellet strikes a solid disk as show in the figure below.  The pellet was traveling at 1 m/s and has a mass of $1$ kg.  The disk has a mass of $10$ kg and radius of 1 m.  If the disk was not initially rotating, what is its final angular velocity after the hit? 

\resizebox{8cm}{!}{\input disk.pdf_t}

\end{document}
