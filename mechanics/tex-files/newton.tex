\documentclass[12pt]{article}
\usepackage[landscape]{geometry}
\usepackage{pdflscape}
\usepackage[pdftex]{graphicx,color} 
\include{def.tex}


\begin{document}
\noindent Two forces $F_1 = 50$ N and $F_2 = 30$ N are exerted on a mass of 30 kg below in the setup below.  Find the acceleration of the mass? 
%\input{vectors/vector_diagram.pdf_t}
\\
\resizebox{8cm}{!}{\input{vector_diagram.pdf_t}}
\newpage
\noindent A 10000 kg locomotive pulls a 20000 kg traincar down a level track.  (a) Draw a free-body diagram of the system. If the acceleration of the train is 0.1 m/s$^2$ (b) find the new force acting on the locomotive-traincar system, (c) the net force acting of the locomotive, (d) net force acting on the traincar.
\newpage
\noindent A mass $m=8$ kg hangs from the ceiling supported by wires with tension $T_1$, $T_2$, and $T_3$.  If $\theta_1 = 45^o$ and $\theta_2=30^o$, find the tensions $T_1$, $T_2$, and $T_3$.\\
\resizebox{12cm}{!}{\input{free_body.pdf_t}}
\newpage
\noindent Consider two masses attached with a wire.  What is the acceleration of the system?  What is the tension of the wire?\\
\resizebox{12cm}{!}{\input{drag.pdf_t}}
\newpage
\noindent A 100 kg box is on a level floor where the coefficient of static friction is $\mu_s = 0.8$.  What force must be exert horizontally to move the box? \\
\newpage
\noindent Consider the plane and pulley system below.   Find the minimum mass for which the more massive block starts sliding up the plane if the coefficient of static friction is $\mu_s = 1.0$. 

\resizebox{7cm}{!}{\input plane2.pdf_t}

\newpage
\noindent A car banks around a 30 degree turn with a 200 m radius. If the track is frictionless what speed should the car approach the turn?

\newpage
\noindent A student stirs a large coffee cup with radius 0.1 m at the rim.  The level of the liquid is 1 cm below the rim of the cup.  What speed does the student have the stir the cup to get the liquid to overflow the rim of the cup?

\end{document}
